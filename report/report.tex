\documentclass{report}
\usepackage[utf8]{inputenc}
\usepackage{graphicx}
\usepackage{titlesec}
\usepackage[tmargin=2.0cm, lmargin=4cm, rmargin=4cm]{geometry}
\usepackage{listings}             % Include the listings-package
\usepackage{textcomp}
\usepackage{listings}
%\usepackage{minted}      % (requires -shell-escape)
\usepackage{xcolor}
\usepackage{filecontents}
\usepackage{array}
\usepackage{multirow}
\usepackage{hyperref}
\usepackage{subfig}
\usepackage{booktabs}
\usepackage{float}

\newcolumntype{L}{>{\centering\arraybackslash}m{3cm}}

\lstset{language=Matlab,%
	morekeywords={matlab2tikz},
	keywordstyle=\color{blue},
	morekeywords=[2]{1}, keywordstyle=[2]{\color{black}},
	identifierstyle=\color{black},
	stringstyle=\color{blue},
	commentstyle=\color{orange},
	showstringspaces=false,%without this there will be a symbol in the places where there is a space
	numbers=left,%
	numberstyle={\tiny \color{black}},% size of the numbers
	numbersep=9pt, % this defines how far the numbers are from the text
	emph=[1]{for,end,break},emphstyle=[1]\color{red}, %some words to emphasise
	%emph=[2]{word1,word2}, emphstyle=[2]{style},
}

% global settings
\lstset
{
  captionpos = below,
  frame      = single,
  breaklines          = true,
  %postbreak=\raisebox{0ex}[0ex][0ex]{\ensuremath{\color{red}\hookrightarrow\space}},
  columns    = fullflexible,
  basicstyle = \ttfamily,
}


%\titlespacing*{\chapter}{0pt}{-19pt}{18pt}

\newcommand{\mychapter}[2]{
    \setcounter{chapter}{#1}
    \setcounter{section}{0}
    \chapter*{#2}
    \addcontentsline{toc}{chapter}{#2}
}
% \topskip 50pt
%\titlespacing*{\mychapter}{0cm}{-50pt}{0pt}[0pt]

\begin{document}

\begin{titlepage}
	\newpage
	\thispagestyle{empty}
	\frenchspacing
	\hspace{-0.2cm}
	\includegraphics[height=3.4cm]{sedes}
	\hspace{0.2cm}
	\rule{0.5pt}{3.4cm}
	\hspace{0.2cm}
	\begin{minipage}[b]{8cm}
		\Large{Katholieke\newline Universiteit\newline Leuven}\smallskip\newline
		\large{}\smallskip\newline
		\textbf{Department of\newline Computer Science}\smallskip
	\end{minipage}
	\hspace{\stretch{1}}
	\vspace*{3.2cm}\vfill
	\begin{center}
		\begin{minipage}[t]{\textwidth}
			\begin{center}
				\LARGE{\rm{\textbf{\uppercase{Project}}}}\\
				\Large{\rm{Genetic Algorithms and Evolutionary Computing (B-KUL-H02D1A) }}\\
				\vspace{0.5cm}

			    \large{\textsc{Verbois-Halilovic}}%

			\end{center}
		\end{minipage}
	\end{center}
	\vfill
	\hfill\makebox[8.5cm][l]{%
		\vbox to 7cm{\vfill\noindent
				{\rm \textbf{Sten Verbois (r0680459)}}\\
				{\rm \textbf{Armin Halilovic (r0679689)}}\\[2mm]
				{\rm Academic year 2017-2018}

		}
	}
\end{titlepage}

\newpage
\tableofcontents
\newpage

\mychapter{0}{Introduction}
In this report we discuss our solutions and results for the given tasks. For each task, experiments were ran to evaluate our solutions. Unless stated otherwise, the experiments were executed with a certain set of parameters and functions. This was done so that we would have a consistent basis to compare results on. The parameters were also chosen in order to leave enough room for improvement so that the effects of different methods can be compared, while at the same time reducing variation of experiments that are too short or do too little work. The default parameters and functions are as follows:
\begin{itemize}
	\item number of individuals = 100
	\item maximum number of generations = 250
	\item probability of mutation = 0.05
	\item probability of crossover = 0.95
	\item percentage of elite population = 0.05
	\item subpopulations = 1
	\item loop detection = off
	\item parent selection function = sus
	\item crossover function = cross\_alternating\_edges
	\item mutation function = mut\_inversion
	\item custom stopping criterion = on
	\item custom survivor selection function = off
\end{itemize}
The results shown in tables are the average results of 10 runs. Every experiment is ran 10 times, so that the effects of local optima would be reduced.

The appendix includes tables that contain results of experiments and our code that is relevant to the tasks.

\mychapter{1}{Tasks}

\section{Task 2: Initial experiments}
The impact of the following parameters on the results of the existing genetic algorithm was tested by varying them while keeping the rest of the parameters at their default values:
\begin{itemize}
	\item number of individuals (NIND)
	\item maximum number of generations (MAXGEN)
	\item percentage of the elite population (ELITIST)
	\item probability of crossover (PR\_CROSS)
	\item probability of mutation (PR\_MUT)
	\item local loop removal (LOCALLOOP)
\end{itemize}
The parameter values were chosen so that evenly spread out options from low to high could be tested. The experiments for this task were executed on a subset of the given datasets to keep the tables readable. The datasets range from ones with a small amount of cities to ones with a large amount of cities. The tables for the results of the experiments can be found in appendix~\ref{app:task2}.

\subsection{Individuals}
The mininum path lengths clearly decrease as the number of individuals increases. This is to be expected, as a larger amount of individuals causes a larger amount of mutations and crossover which can positively impact path lengths. Analogously, the maximum path lengths slightly increase as the number of individuals increases. Because of this effect, the mean path lengths remain relatively constant after 100 individuals.

\subsection{Generations}
TODO: COMMENTS

\subsection{Elitism}
TODO: COMMENTS

\subsection{Crossover}
TODO: COMMENTS

\subsection{Mutation}
TODO: COMMENTS

\subsection{Loop removal}
See section~\ref{sec:local}

\subsection{Mix}
After some parameter tuning with the above information in mind, we have come up with a configuration of parameters that performs very well. The results can be seen in table~\ref{tab:todo}. The parameters were:



\section{Task 3: Stopping criterion}
To implement a new stopping criterion, we looked at the commonly used termination conditions outlined by the textbook. There we see the following suggestions:
\begin{enumerate}
	\item Maximally allowed CPU time elapses.
	\item Total number of fitness evaluations reaches limit.
	\item Fitness improvement remains under threshold for a given period of time.
	\item Population diversity drops under threshold.
\end{enumerate}
The first and second criteria are useful, either to guarantee the evaluations do not go on forever, or when there is some kind of constraint on system resource usage. In the project template, we already have the guarantee of eventual termination because of the limit on the number of generations, and we do not have to account for system resource constraints.

The fourth criterion is also already present in the template and can be adjusted via the GUI. The default value is so strict (95\% equal individuals), it practically is never reached.

We decided to implement the third criterion. With this condition, termination occurs when the fitness of the best individual does not improve above a threshold for a given period of time. This period of time is expressed in terms of a certain number of generations. We chose to define this number of generations to be a percentage of the specified maximum number of generations. When testing this termination condition, we see that it does succeed in avoiding computation of useless generations where the fitness does not improve for a long time. Because of the fact that improvements may still happen at a later point in time, the score will be slightly worse with this new condition.

The results of our experiments with this new termination condition are displayed in Table~\ref{tab:customstop}.

\begin{table}[H] 
\centering 
\makebox[\textwidth]{
\begin{tabular}{l rrrr c rrrr} 
\toprule
& \multicolumn{4}{c}{Default stopping criterion} & \phantom{abc} & \multicolumn{4}{c}{Custom stopping criterion}\\
\cmidrule{2-5} \cmidrule{7-10}
Dataset & \# Generations & Min & Mean & Max && \# Generations & Min & Mean & Max\\ 
\midrule
rondrit016.tsp & 184.0 & 3.3752 & 3.4664 & 4.4159 && 75.2 & 3.5061 & 4.6191 & 6.1368 \\
rondrit018.tsp & 232.4 & 3.0469 & 4.3285 & 5.8945 && 92.6 & 3.3740 & 5.1082 & 7.1152 \\
rondrit023.tsp & 250.0 & 4.0076 & 6.6840 & 9.6150 && 84.3 & 4.5584 & 7.0113 & 9.6520 \\
rondrit025.tsp & 250.0 & 5.0209 & 8.7816 & 12.2293 && 113.0 & 5.7750 & 9.1623 & 12.6250 \\
rondrit048.tsp & 250.0 & 9.6163 & 14.4618 & 18.6610 && 83.5 & 10.6749 & 15.1442 & 19.1976 \\
rondrit050.tsp & 250.0 & 13.9410 & 19.4818 & 24.0982 && 120.1 & 15.2353 & 20.3705 & 24.6610 \\
rondrit051.tsp & 250.0 & 13.2179 & 18.3265 & 22.4049 && 90.3 & 14.7031 & 19.0930 & 23.3291 \\
rondrit067.tsp & 250.0 & 13.6788 & 18.6416 & 22.5891 && 136.5 & 14.3745 & 18.7647 & 22.5262 \\
rondrit070.tsp & 250.0 & 21.7930 & 28.4651 & 34.0860 && 99.3 & 23.3636 & 29.3783 & 34.7131 \\
rondrit100.tsp & 250.0 & 34.1562 & 43.1992 & 50.1761 && 100.6 & 36.6088 & 44.4320 & 50.6472 \\
rondrit127.tsp & 250.0 & 22.0104 & 26.3718 & 29.7679 && 119.6 & 22.8799 & 26.7850 & 30.2722 \\
\bottomrule 
\end{tabular} 
}
\caption{Comparison between default and custom stopping criteria.}
\label{tab:customstop}
\end{table}


\section{Task 4: Other representation}
The given project template uses adjacency representation by default for TSP paths. We have chosen to use path representation as the alternative one. Conversion between the two representations was already possible thanks to the 'adj2path' and 'path2adj' functions in the template. To do crossover with path representation, we implemented the Order Crossover method (function 'cross\_order') as described in the textbook. Simple Inversion Mutation, which is a mutation operator for path representation, was already provided in the template ('mut\_inversion'). We have decided to extend this and have added a function for Inversion Mutation ('mut\_inversion2').

Table~\ref{tab:path_repr_crossover} contains the results of experiments with different crossover operators. The 'cross\_alternating\_edges' function implements Alternating Edge Crossover and is provided in the template. It is clear that Order Crossover performs significantly better than Alternating Edge Crossover; all of the path lengths with Order Crossover are lower for every dataset.
\begin{table}[H] 
\centering 
\makebox[\textwidth]{
\begin{tabular}{l rrrr c rrrr} 
\toprule
& \multicolumn{4}{c}{Alternating Edge Crossover} & \phantom{abc} & \multicolumn{4}{c}{Order Crossover}\\
\cmidrule{2-5} \cmidrule{7-10}
Dataset & \# Generations & Min & Mean & Max && \# Generations & Min & Mean & Max\\ 
\midrule
rondrit016.tsp & 182.4 & 3.39 & 3.55 & 4.32 && 50.6 & 3.44 & 3.46 & 4.08 \\
rondrit018.tsp & 244.5 & 2.98 & 4.53 & 6.46 && 55.4 & 3.05 & 3.06 & 3.87 \\
rondrit023.tsp & 250.0 & 3.90 & 6.59 & 9.43 && 79.0 & 3.57 & 3.59 & 4.45 \\
rondrit025.tsp & 250.0 & 5.03 & 8.64 & 11.94 && 82.4 & 4.48 & 4.51 & 5.75 \\
rondrit048.tsp & 250.0 & 9.68 & 14.45 & 18.81 && 188.4 & 5.49 & 5.54 & 6.67 \\
rondrit050.tsp & 250.0 & 13.99 & 19.79 & 24.57 && 162.4 & 8.20 & 8.23 & 9.53 \\
rondrit051.tsp & 250.0 & 13.08 & 18.29 & 22.54 && 155.9 & 8.67 & 8.69 & 9.61 \\
rondrit067.tsp & 250.0 & 13.95 & 18.41 & 22.28 && 188.9 & 7.11 & 7.14 & 7.92 \\
rondrit070.tsp & 250.0 & 21.89 & 28.98 & 34.40 && 219.6 & 10.94 & 11.06 & 12.49 \\
rondrit100.tsp & 250.0 & 34.40 & 43.21 & 50.05 && 232.6 & 17.78 & 18.22 & 19.91 \\
rondrit127.tsp & 250.0 & 21.93 & 26.30 & 29.49 && 239.8 & 12.77 & 13.03 & 14.02 \\
\bottomrule 
\end{tabular} 
}
\caption{Results of different crossover functions.}
\label{tab:path_repr_crossover}
\end{table}


Table~\ref{tab:path_repr_mutation} contains the results of experiments with different mutation operators. The 'mut\_inversion' function implements Simple Inversion Mutation and is provided in the template. There is no clear difference between the two methods in these results.
\begin{table}[H] 
\centering 
\makebox[\textwidth]{
\begin{tabular}{l rrrr c rrrr} 
\toprule
& \multicolumn{4}{c}{Simple Inversion Mutation} & \phantom{abc} & \multicolumn{4}{c}{Inversion Mutation}\\
\cmidrule{2-5} \cmidrule{7-10}
Dataset & \# Generations & Min & Mean & Max && \# Generations & Min & Mean & Max\\ 
\midrule
rondrit016.tsp & 80.7 & 1561.14 & 2033.04 & 2755.93 && 70.9 & 1544.57 & 2098.83 & 2882.41 \\
rondrit018.tsp & 108.9 & 1285.59 & 1962.10 & 2742.77 && 117.0 & 1280.33 & 2017.12 & 2852.09 \\
rondrit023.tsp & 95.8 & 1059.03 & 1690.34 & 2353.04 && 101.2 & 1063.86 & 1667.81 & 2331.48 \\
rondrit025.tsp & 121.7 & 55.85 & 90.13 & 125.32 && 115.5 & 56.13 & 89.57 & 120.63 \\
rondrit048.tsp & 98.4 & 8197.14 & 11667.37 & 14942.97 && 103.6 & 8217.13 & 11840.02 & 15084.46 \\
rondrit050.tsp & 96.7 & 15.25 & 20.15 & 24.51 && 94.8 & 15.46 & 20.31 & 24.36 \\
rondrit051.tsp & 114.9 & 998.05 & 1308.72 & 1581.53 && 99.1 & 1019.56 & 1318.60 & 1598.90 \\
rondrit067.tsp & 98.6 & 7533.35 & 9485.01 & 11261.62 && 86.6 & 7576.08 & 9504.70 & 11326.00 \\
rondrit070.tsp & 128.8 & 2265.73 & 2939.31 & 3421.83 && 107.3 & 2312.05 & 2961.56 & 3430.63 \\
rondrit100.tsp & 98.3 & 37.08 & 44.17 & 50.75 && 106.1 & 36.25 & 44.02 & 50.39 \\
rondrit127.tsp & 119.3 & 467.69 & 544.58 & 608.25 && 102.0 & 466.99 & 540.11 & 606.35 \\
\bottomrule 
\end{tabular} 
}
\caption{Results of different mutation functions.}
\label{tab:path_repr_mutation}
\end{table}


Table~\ref{tab:path_repr_tuning} shows results after some parameter tuning with the 'cross\_order' and 'mut\_inversion2' operators. The parameters used were: a = b, c = d, ... TODO  check tuning
\begin{table}[H] 
\centering 
\makebox[\textwidth]{
\begin{tabular}{l rrrr} 
\toprule
Dataset & \# Generations & Min & Mean & Max\\ 
\midrule
rondrit016.tsp & 66.9 & 1492.80 & 1540.95 & 1998.85 \\
rondrit018.tsp & 87.6 & 1226.72 & 1253.20 & 1732.80 \\
rondrit023.tsp & 115.8 & 841.06 & 873.15 & 1268.13 \\
rondrit025.tsp & 126.0 & 42.69 & 43.37 & 61.85 \\
rondrit048.tsp & 195.8 & 4460.91 & 4518.69 & 5778.26 \\
rondrit050.tsp & 212.6 & 7.65 & 8.05 & 10.16 \\
rondrit051.tsp & 182.8 & 610.21 & 614.13 & 709.88 \\
rondrit067.tsp & 228.2 & 3456.49 & 3613.75 & 4305.84 \\
rondrit070.tsp & 224.4 & 1163.05 & 1191.41 & 1390.02 \\
rondrit100.tsp & 250.0 & 17.71 & 18.18 & 20.04 \\
rondrit127.tsp & 250.0 & 257.65 & 262.61 & 282.46 \\
\bottomrule 
\end{tabular} 
}
\caption{Results for operators for path representation after parameter tuning.}
\label{tab:path_repr_tuning}
\end{table}




\section{Task 5: Local optimisation}
\label{sec:local}
For this task, we are testing the local optimization already present in the template. This optimization takes a path and tries to remove local loops up to path length 3. With default values for other parameters, this results in major improvements to the score.

The results of our experiments with local optimisation disabled and enabled are displayed in Table~\ref{tab:localopt}.

\begin{table}[H] 
\centering 
\makebox[\textwidth]{
\begin{tabular}{l rrrr c rrrr} 
\toprule
& \multicolumn{4}{c}{Local optimisation disabled} & \phantom{abc} & \multicolumn{4}{c}{Local optimisation enabled}\\
\cmidrule{2-5} \cmidrule{7-10}
Dataset & \# Generations & Min & Mean & Max && \# Generations & Min & Mean & Max\\ 
\midrule
rondrit016.tsp & 172.3 & 3.7123 & 3.8700 & 4.8527 && 41.3 & 3.6882 & 3.7103 & 4.5303 \\
rondrit018.tsp & 268.5 & 3.3213 & 4.7075 & 6.6602 && 80.2 & 3.2205 & 3.2420 & 4.0900 \\
rondrit023.tsp & 275.0 & 4.3712 & 7.3143 & 10.2788 && 196.0 & 3.5667 & 4.5078 & 6.4752 \\
rondrit025.tsp & 275.0 & 5.6995 & 9.6434 & 13.7563 && 258.5 & 4.4518 & 7.0959 & 10.2260 \\
rondrit048.tsp & 275.0 & 10.4986 & 16.0656 & 21.0702 && 275.0 & 6.5498 & 11.6610 & 16.0820 \\
rondrit050.tsp & 275.0 & 15.1079 & 21.5223 & 26.5500 && 275.0 & 9.5271 & 15.6665 & 20.5584 \\
rondrit051.tsp & 275.0 & 14.8153 & 20.3610 & 24.8127 && 275.0 & 9.0674 & 14.5821 & 18.9597 \\
rondrit067.tsp & 275.0 & 15.1277 & 20.2642 & 24.3336 && 275.0 & 8.9199 & 14.4614 & 18.3117 \\
rondrit070.tsp & 275.0 & 24.1041 & 31.6820 & 38.3154 && 275.0 & 13.6850 & 21.9450 & 27.7873 \\
rondrit100.tsp & 275.0 & 37.9887 & 47.3694 & 55.1080 && 275.0 & 21.1937 & 32.3180 & 39.5336 \\
rondrit127.tsp & 275.0 & 23.9401 & 28.9510 & 32.9979 && 275.0 & 14.2627 & 20.2005 & 24.0442 \\
\bottomrule 
\end{tabular} 
}
\caption{Comparison between local optimisation disabled (left) and local optimisation enabled (right).}
\label{tab:localopt}
\end{table}



\section{Task 7: Optional tasks}
\subsection{7a: Parent selection}
Additional parent selection methods we implemented are Fitness Proportional Selection ('sel\_fit\_prop') and Tournament Selection ('sel\_tournament'). Both of them use the same parameters as the existing implementation of Stochastic Universal Sampling so we could easily swap them in.

\begin{table}[H]
	\centering
	\makebox[\textwidth]{
		\begin{tabular}{l rrrr}
			\toprule
			& \multicolumn{4}{c}{Stochastic Universal Sampling} \\
			\cmidrule{2-5}
			Dataset & \# Generations & Min & Mean & Max \\
			\midrule
rondrit016.tsp & 76.5 & 3.5071 & 4.7287 & 6.29733152234251 \\
rondrit018.tsp & 112.0 & 3.2458 & 4.9674 & 6.91548211300174 \\
rondrit023.tsp & 89.8 & 4.3998 & 6.9255 & 9.64269815478443 \\
rondrit025.tsp & 81.6 & 5.8037 & 9.2203 & 12.46022806183646 \\
rondrit048.tsp & 115.7 & 10.5298 & 15.0273 & 18.93401510803271 \\
rondrit050.tsp & 74.1 & 15.5617 & 20.4526 & 24.58418303604441 \\
rondrit051.tsp & 101.3 & 14.7763 & 19.2497 & 23.21947103058057 \\
rondrit067.tsp & 98.2 & 14.7791 & 19.1254 & 22.65609716605926 \\
rondrit070.tsp & 117.4 & 22.9912 & 29.3627 & 34.87830719868566 \\
rondrit100.tsp & 91.4 & 37.0343 & 44.2892 & 51.06123059198230 \\
rondrit127.tsp & 120.9 & 22.9939 & 26.6807 & 30.08382590490890 \\
			\bottomrule
		\end{tabular}
	}
	\caption{Results when using Stochastic Universal Sampling as parent selection method.}
	\label{tab:selparentsus}
\end{table}

\begin{table}[H]
	\centering
	\makebox[\textwidth]{
		\begin{tabular}{l rrrr}
			\toprule
			& \multicolumn{4}{c}{Tournament Selection} \\
			\cmidrule{2-5}
			Dataset & \# Generations & Min & Mean & Max \\
			\midrule
rondrit016.tsp & 47.6 & 3.4307 & 3.4489 & 4.11682588428370 \\
rondrit018.tsp & 69.8 & 3.0751 & 3.1000 & 3.94962726552862 \\
rondrit023.tsp & 97.2 & 3.4223 & 3.4448 & 4.36636737136705 \\
rondrit025.tsp & 107.1 & 4.2080 & 4.2289 & 5.22930916811214 \\
rondrit048.tsp & 220.2 & 5.2384 & 5.3288 & 6.70164030513469 \\
rondrit050.tsp & 228.5 & 7.4060 & 7.4469 & 8.74950293351244 \\
rondrit051.tsp & 223.4 & 7.7256 & 7.8091 & 9.00104215733518 \\
rondrit067.tsp & 246.7 & 6.5806 & 6.6540 & 7.79291635925291 \\
rondrit070.tsp & 250.0 & 10.4302 & 11.1550 & 14.05669608369397 \\
rondrit100.tsp & 241.3 & 18.2543 & 19.2141 & 22.21694990667605 \\
rondrit127.tsp & 250.0 & 13.2633 & 13.8539 & 15.84629676598155 \\
			\bottomrule
		\end{tabular}
	}
	\caption{Results when using Tournament Selection as parent selection method.}
	\label{tab:selparenttournament}
\end{table}

\begin{table}[H]
	\centering
	\makebox[\textwidth]{
		\begin{tabular}{l rrrr}
			\toprule
			& \multicolumn{4}{c}{Fitness Proportional Selection} \\
			\cmidrule{2-5}
			Dataset & \# Generations & Min & Mean & Max \\
			\midrule
rondrit016.tsp & 87.6 & 3.5253 & 4.7702 & 6.31735983363770 \\
rondrit018.tsp & 106.5 & 3.1629 & 5.0059 & 6.88030465197992 \\
rondrit023.tsp & 72.1 & 4.6849 & 7.0675 & 9.61113827405700 \\
rondrit025.tsp & 115.5 & 5.7095 & 9.0493 & 12.57024535746879 \\
rondrit048.tsp & 93.0 & 10.6731 & 15.1775 & 19.06378066242915 \\
rondrit050.tsp & 84.1 & 15.6907 & 20.4983 & 24.46309616398185 \\
rondrit051.tsp & 114.1 & 14.6637 & 19.0878 & 22.88529001223297 \\
rondrit067.tsp & 101.7 & 14.8841 & 19.0997 & 22.56887763300833 \\
rondrit070.tsp & 101.4 & 23.2645 & 29.7146 & 35.05263230223609 \\
rondrit100.tsp & 95.0 & 37.2210 & 43.9292 & 50.80513760208527 \\
rondrit127.tsp & 114.7 & 22.9952 & 26.8508 & 29.97739386912141 \\
			\bottomrule
		\end{tabular}
	}
	\caption{Results when using Fitness Proportional Selection as parent selection method.}
	\label{tab:selparentfitprop}
\end{table}

\subsection{7b: Survivor selection}
Round robin tournament was chosen as the other strategy for survivor selection. The results for different values of the ELITISM parameter can be found in table~\ref{tab:vary_elitism}. To evaluate how round robin tournament performs compared to the already implemented elitism, the ELITIST parameter is set to 0 in an experiment. Also, an experiment is done where elitism is combined with round robin tournament. Table~\ref{tab:survivor_selection} contains the results of the experiments for this task. TODO: evaluate results after verifying they are correct. The custom stopping criterion needs to be updated first. Last X values should all be equal, instead of just current and current-X.
\begin{table}[H] 
\centering 
\makebox[\textwidth]{
\begin{tabular}{l rrrr c rrrr} 
\toprule
& \multicolumn{4}{c}{Already implemented elitism} & \phantom{abc} & \multicolumn{4}{c}{Round robin tournament}\\
\cmidrule{2-5} \cmidrule{7-10}
Dataset & \# Generations & Min & Mean & Max && \# Generations & Min & Mean & Max\\ 
\midrule
rondrit016.tsp & 91.3 & 3.46 & 4.66 & 6.19 && 0.0 & 0.00 & 0.00 & 0.00 \\
rondrit018.tsp & 83.9 & 3.40 & 5.20 & 7.09 && 0.0 & 0.00 & 0.00 & 0.00 \\
rondrit023.tsp & 84.3 & 4.50 & 6.91 & 9.45 && 0.0 & 0.00 & 0.00 & 0.00 \\
rondrit025.tsp & 102.9 & 5.79 & 9.11 & 12.60 && 0.0 & 0.00 & 0.00 & 0.00 \\
rondrit048.tsp & 101.0 & 10.56 & 15.16 & 19.12 && 0.0 & 0.00 & 0.00 & 0.00 \\
rondrit050.tsp & 137.6 & 14.68 & 19.80 & 24.21 && 0.0 & 0.00 & 0.00 & 0.00 \\
rondrit051.tsp & 111.3 & 14.70 & 19.24 & 22.88 && 0.0 & 0.00 & 0.00 & 0.00 \\
rondrit067.tsp & 90.4 & 15.17 & 19.11 & 22.88 && 0.0 & 0.00 & 0.00 & 0.00 \\
rondrit070.tsp & 123.4 & 23.12 & 29.61 & 35.04 && 0.0 & 0.00 & 0.00 & 0.00 \\
rondrit100.tsp & 106.5 & 36.29 & 43.94 & 50.20 && 0.0 & 0.00 & 0.00 & 0.00 \\
rondrit127.tsp & 113.6 & 23.02 & 26.88 & 30.19 && 0.0 & 0.00 & 0.00 & 0.00 \\
\bottomrule 
\end{tabular} 
}
\caption{Results for the already implemented elitism and our round robin tournament survivor selection.}
\label{tab:survivor_selection}
\end{table}


\subsection{7c: Diversity preservation}
In order to preserve population diversity, we adapted a few of the functions in the template to work with subpopulations, simulating the island model. The results displayed in Table~\ref{tab:diversity1} through Table~\ref{tab:diversity20} show tests performed with 1, 2, 5, 10 and 20 subpopulations or islands.

\begin{table}[H]
	\centering
	\makebox[\textwidth]{
		\begin{tabular}{l rrrr}
			\toprule
			& \multicolumn{4}{c}{\# subpopulations = 1} \\
			\cmidrule{2-5}
			Dataset & \# Generations & Min & Mean & Max \\
			\midrule
			rondrit016.tsp & 99.9 & 3.7370 & 5.3014 & 7.3443 \\
			rondrit018.tsp & 120.7 & 3.4504 & 5.5736 & 7.9594 \\
			rondrit023.tsp & 124.4 & 4.6149 & 7.6039 & 10.7075 \\
			rondrit025.tsp & 134.5 & 5.8402 & 9.9015 & 13.9843 \\
			rondrit048.tsp & 129.7 & 10.9175 & 16.1203 & 21.3953 \\
			rondrit050.tsp & 194.6 & 14.7423 & 21.4762 & 26.9975 \\
			rondrit051.tsp & 169.1 & 14.7132 & 20.3833 & 25.4901 \\
			rondrit067.tsp & 139.0 & 15.3320 & 20.6747 & 25.3712 \\
			rondrit070.tsp & 187.9 & 23.1837 & 31.6961 & 39.3717 \\
			rondrit100.tsp & 174.5 & 38.1144 & 48.0221 & 55.8977 \\
			rondrit127.tsp & 195.0 & 23.8434 & 29.0456 & 33.1707 \\
			\bottomrule
		\end{tabular}
	}
	\caption{Results when using a single subpopulation.}
	\label{tab:diversity1}
\end{table}

\begin{table}[H]
	\centering
	\makebox[\textwidth]{
		\begin{tabular}{l rrrr}
			\toprule
			& \multicolumn{4}{c}{\# subpopulations = 2} \\
			\cmidrule{2-5}
			Dataset & \# Generations & Min & Mean & Max \\
			\midrule
			rondrit016.tsp & 118.9 & 4.1112 & 5.6874 & 7.9026 \\
			rondrit018.tsp & 147.5 & 3.8393 & 6.0843 & 8.5831 \\
			rondrit023.tsp & 159.4 & 5.0312 & 8.3282 & 11.8303 \\
			rondrit025.tsp & 152.9 & 6.6344 & 10.9143 & 15.5315 \\
			rondrit048.tsp & 204.9 & 11.4129 & 17.6468 & 23.8143 \\
			rondrit050.tsp & 216.0 & 16.5176 & 23.7049 & 29.4113 \\
			rondrit051.tsp & 168.2 & 16.6110 & 22.5015 & 27.8662 \\
			rondrit067.tsp & 203.0 & 16.4217 & 22.3292 & 27.4292 \\
			rondrit070.tsp & 129.3 & 27.3049 & 35.4823 & 42.9792 \\
			rondrit100.tsp & 251.9 & 40.3992 & 51.8574 & 60.8410 \\
			rondrit127.tsp & 265.4 & 25.6029 & 31.5892 & 36.3884 \\
			\bottomrule
		\end{tabular}
	}
	\caption{Results when using two subpopulations.}
	\label{tab:diversity2}
\end{table}

\begin{table}[H]
	\centering
	\makebox[\textwidth]{
		\begin{tabular}{l rrrr}
			\toprule
			& \multicolumn{4}{c}{\# subpopulations = 5} \\
			\cmidrule{2-5}
			Dataset & \# Generations & Min & Mean & Max \\
			\midrule
			rondrit016.tsp & 126.6 & 4.1295 & 5.8593 & 7.9499 \\
			rondrit018.tsp & 121.0 & 4.0043 & 6.2663 & 8.9579 \\
			rondrit023.tsp & 138.2 & 5.1783 & 8.4623 & 12.0132 \\
			rondrit025.tsp & 166.0 & 6.4178 & 11.0127 & 15.7281 \\
			rondrit048.tsp & 117.9 & 12.4779 & 18.4335 & 24.2895 \\
			rondrit050.tsp & 164.8 & 17.1913 & 24.2682 & 30.5998 \\
			rondrit051.tsp & 166.4 & 17.3803 & 23.2727 & 28.6140 \\
			rondrit067.tsp & 225.4 & 16.5562 & 22.6888 & 27.6837 \\
			rondrit070.tsp & 183.2 & 26.8295 & 35.8003 & 42.9375 \\
			rondrit100.tsp & 199.2 & 41.8170 & 53.1547 & 62.3695 \\
			rondrit127.tsp & 218.5 & 26.4587 & 32.1475 & 36.5978 \\
			\bottomrule
		\end{tabular}
	}
	\caption{Results when using five subpopulations.}
	\label{tab:diversity5}
\end{table}

\begin{table}[H]
	\centering
	\makebox[\textwidth]{
		\begin{tabular}{l rrrr}
			\toprule
			& \multicolumn{4}{c}{\# subpopulations = 10} \\
			\cmidrule{2-5}
			Dataset & \# Generations & Min & Mean & Max \\
			\midrule
			rondrit016.tsp & 97.4 & 4.1405 & 5.5019 & 7.7501 \\
			rondrit018.tsp & 140.1 & 3.8233 & 5.5875 & 8.1174 \\
			rondrit023.tsp & 180.4 & 4.7633 & 7.8333 & 11.6950 \\
			rondrit025.tsp & 186.0 & 6.1102 & 10.4032 & 15.4008 \\
			rondrit048.tsp & 276.3 & 10.6240 & 17.0398 & 23.1415 \\
			rondrit050.tsp & 245.8 & 15.9304 & 22.8300 & 29.9988 \\
			rondrit051.tsp & 190.7 & 16.0860 & 21.8967 & 27.5143 \\
			rondrit067.tsp & 254.4 & 15.8431 & 21.9523 & 27.1892 \\
			rondrit070.tsp & 272.4 & 24.4098 & 33.8360 & 41.4091 \\
			rondrit100.tsp & 250.7 & 39.4631 & 51.1833 & 61.1145 \\
			rondrit127.tsp & 282.1 & 25.3591 & 31.1691 & 36.3321 \\
			\bottomrule
		\end{tabular}
	}
	\caption{Results when using ten subpopulations.}
	\label{tab:diversity10}
\end{table}

\section{Task 6: Benchmark problems}
For this task, we have selected a set of parameters and methods based on all of the results above. Our algorithm is evaluated by running it on given benchmark problems and calculating the relative error of the results of the algorithm to the known optimal paths of the benchmark problems.
The parameters and methods used were as follows:
\begin{itemize}
	\item TODO
\end{itemize}
The results are shown in table~\ref{tab:benchmark}. Judging from the relative errors, we can conclude that our solutions are TODO
\begin{table}[H] 
\centering 
\makebox[\textwidth]{
\begin{tabular}{l rrrrr c rrr} 
\toprule
Dataset & Optimal length & \# Generations & Min & Mean & Max && Error Min & Error Mean & Error Max\\ 
\midrule
bcl380.tsp & 1621 & 1000.0 & 1896.69 & 1904.51 & 2051.59 && 17.01\% & 17.49\% & 126.50\% \\
belgiumtour.tsp & 1 & 1000.0 & 682.82 & 688.46 & 825.21 && 68182.28\% & 68746.22\% & 82421.11\% \\
rbx711.tsp & 3115 & 1000.0 & 4472.28 & 4481.97 & 4688.02 && 43.57\% & 43.88\% & 150.47\% \\
xqf131.tsp & 564 & 1000.0 & 598.65 & 601.90 & 663.49 && 6.14\% & 6.72\% & 117.46\% \\
xql662.tsp & 2513 & 1000.0 & 3535.07 & 3538.77 & 3692.50 && 40.67\% & 40.82\% & 146.90\% \\
\bottomrule 
\end{tabular} 
}
\caption{Results for benchmark problems with our final algorithm.}
\label{tab:benchmark}
\end{table}

parameters:
NIND=1000;           % Number of individuals
MAXGEN=1000;         % Maximum no. of generations
ELITIST=0;           % percentage of the elite population
STOP_PERCENTAGE=.95; % percentage of equal fitness individuals for stopping
PR_CROSS=.35;        % probability of crossover
PR_MUT=.50;          % probability of mutation
LOCALLOOP=1;         % local loop removal
CROSSOVER = 'cross_order'; % crossover operators
MUTATION = 'mut_inversion'; % mutation operators
SELECTION = 'sel_tournament'; % parent selection algorithm
SUBPOP = 1;          % Amount of subpopulations
CUSTOMSTOP = 0;      % Custom stopping criterion on/off
CUSTOMSS = 1;        % Custom survivor selection on/off
RUNS = 5;            % Number of ga runs in tests

\mychapter{4}{Appendix}
\section{Tables}
\subsection{Task 2}
\label{app:task2}
\begin{table}[H] 
\centering 
\makebox[\textwidth]{
\begin{tabular}{l rrrr} 
\toprule
Dataset & \# Generations & Min & Mean & Max\\ 
\midrule
\multicolumn{5}{c}{number of individuals = 50}\\ 
\midrule

\begin{table}[H] 
\centering 
\makebox[\textwidth]{
\begin{tabular}{l rrrr} 
\toprule
Dataset & \# Generations & Min & Mean & Max\\ 
\midrule
\multicolumn{5}{c}{max number of generations = 100}\\ 
\midrule
rondrit016.tsp & 110.0 & 3.96 & 5.59 & 7.16 \\
rondrit048.tsp & 110.0 & 12.55 & 17.37 & 21.65 \\
rondrit067.tsp & 110.0 & 17.23 & 21.73 & 25.31 \\
rondrit127.tsp & 110.0 & 26.64 & 30.43 & 33.49 \\
\midrule
\multicolumn{5}{c}{max number of generations = 250}\\ 
\midrule
rondrit016.tsp & 266.0 & 3.78 & 5.11 & 6.40 \\
rondrit048.tsp & 275.0 & 11.76 & 17.25 & 21.25 \\
rondrit067.tsp & 275.0 & 16.49 & 21.51 & 24.93 \\
rondrit127.tsp & 275.0 & 25.72 & 29.78 & 32.79 \\
\midrule
\multicolumn{5}{c}{max number of generations = 500}\\ 
\midrule
rondrit016.tsp & 455.4 & 3.74 & 4.75 & 5.79 \\
rondrit048.tsp & 550.0 & 11.36 & 16.91 & 20.90 \\
rondrit067.tsp & 550.0 & 15.73 & 21.11 & 24.53 \\
rondrit127.tsp & 550.0 & 24.90 & 29.81 & 32.72 \\
\midrule
\multicolumn{5}{c}{max number of generations = 1000}\\ 
\midrule
rondrit016.tsp & 785.0 & 3.75 & 4.54 & 5.49 \\
rondrit048.tsp & 1100.0 & 10.44 & 16.51 & 20.78 \\
rondrit067.tsp & 1100.0 & 14.92 & 20.82 & 24.94 \\
rondrit127.tsp & 1100.0 & 23.76 & 29.38 & 32.65 \\
\bottomrule 
\end{tabular} 
}
\caption{Existing genetic algorithm with varying amount of maximum generations.}
\label{tab:vary_gen}
\end{table}

\begin{table}[H] 
\centering 
\makebox[\textwidth]{
\begin{tabular}{l rrrr} 
\toprule
Dataset & \# Generations & Min & Mean & Max\\ 
\midrule
\multicolumn{5}{c}{percentage of the elite population = 0.00}\\ 
\midrule
rondrit016.tsp & 110.0 & 5.15 & 6.30 & 7.58 \\
rondrit048.tsp & 110.0 & 15.75 & 19.09 & 22.47 \\
rondrit067.tsp & 110.0 & 20.23 & 23.38 & 26.18 \\
rondrit127.tsp & 110.0 & 29.17 & 31.70 & 34.01 \\
\midrule
\multicolumn{5}{c}{percentage of the elite population = 0.05}\\ 
\midrule
rondrit016.tsp & 110.0 & 3.94 & 5.65 & 7.21 \\
rondrit048.tsp & 110.0 & 12.71 & 17.51 & 20.96 \\
rondrit067.tsp & 110.0 & 17.35 & 21.71 & 25.04 \\
rondrit127.tsp & 110.0 & 26.47 & 30.39 & 33.17 \\
\midrule
\multicolumn{5}{c}{percentage of the elite population = 0.15}\\ 
\midrule
rondrit016.tsp & 55.4 & 3.84 & 3.86 & 4.50 \\
rondrit048.tsp & 108.3 & 9.98 & 11.85 & 15.87 \\
rondrit067.tsp & 110.0 & 14.41 & 16.05 & 19.57 \\
rondrit127.tsp & 110.0 & 24.11 & 26.47 & 30.17 \\
\midrule
\multicolumn{5}{c}{percentage of the elite population = 0.35}\\ 
\midrule
rondrit016.tsp & 56.2 & 3.84 & 3.86 & 4.45 \\
rondrit048.tsp & 108.0 & 9.99 & 11.09 & 14.44 \\
rondrit067.tsp & 107.4 & 14.27 & 15.48 & 18.45 \\
rondrit127.tsp & 110.0 & 23.16 & 24.46 & 27.44 \\
\midrule
\multicolumn{5}{c}{percentage of the elite population = 0.50}\\ 
\midrule
rondrit016.tsp & 68.0 & 3.83 & 3.84 & 4.11 \\
rondrit048.tsp & 108.6 & 10.08 & 10.96 & 14.69 \\
rondrit067.tsp & 110.0 & 14.75 & 15.78 & 18.37 \\
rondrit127.tsp & 110.0 & 23.85 & 25.07 & 28.28 \\
\midrule
\multicolumn{5}{c}{percentage of the elite population = 0.75}\\ 
\midrule
rondrit016.tsp & 93.7 & 3.92 & 3.97 & 4.47 \\
rondrit048.tsp & 110.0 & 11.29 & 12.24 & 15.04 \\
rondrit067.tsp & 106.0 & 16.31 & 17.05 & 19.70 \\
rondrit127.tsp & 110.0 & 24.89 & 25.72 & 28.15 \\
\midrule
\multicolumn{5}{c}{percentage of the elite population = 0.95}\\ 
\midrule
rondrit016.tsp & 110.0 & 4.75 & 5.12 & 5.68 \\
rondrit048.tsp & 110.0 & 14.80 & 16.06 & 17.67 \\
rondrit067.tsp & 110.0 & 19.33 & 20.71 & 22.18 \\
rondrit127.tsp & 110.0 & 28.49 & 29.52 & 30.69 \\
\bottomrule 
\end{tabular} 
}
\caption{Existing genetic algorithm with varying percentage of the elite population.}
\label{tab:vary_elitism}
\end{table}

\begin{table}[H] 
\centering 
\makebox[\textwidth]{
\begin{tabular}{l rrrr} 
\toprule
Dataset & \# Generations & Min & Mean & Max\\ 
\midrule
\multicolumn{5}{c}{probability of crossover = 0.00}\\ 
\midrule
rondrit016.tsp & 31.5 & 5.14 & 5.15 & 5.47 \\
rondrit048.tsp & 83.4 & 13.74 & 13.75 & 14.18 \\
rondrit067.tsp & 90.9 & 18.62 & 18.63 & 18.92 \\
rondrit127.tsp & 141.2 & 26.71 & 26.71 & 26.90 \\
\midrule
\multicolumn{5}{c}{probability of crossover = 0.10}\\ 
\midrule
rondrit016.tsp & 37.0 & 4.41 & 4.42 & 4.83 \\
rondrit048.tsp & 120.0 & 10.21 & 10.23 & 10.79 \\
rondrit067.tsp & 151.2 & 14.42 & 14.43 & 14.94 \\
rondrit127.tsp & 199.4 & 22.26 & 22.30 & 22.58 \\
\midrule
\multicolumn{5}{c}{probability of crossover = 0.30}\\ 
\midrule
rondrit016.tsp & 40.9 & 4.16 & 4.17 & 4.67 \\
rondrit048.tsp & 163.6 & 8.41 & 8.44 & 9.43 \\
rondrit067.tsp & 207.7 & 11.44 & 11.60 & 12.54 \\
rondrit127.tsp & 259.8 & 18.98 & 19.29 & 20.44 \\
\midrule
\multicolumn{5}{c}{probability of crossover = 0.50}\\ 
\midrule
rondrit016.tsp & 46.8 & 4.07 & 4.07 & 4.44 \\
rondrit048.tsp & 171.2 & 8.09 & 8.15 & 9.11 \\
rondrit067.tsp & 246.2 & 10.36 & 10.57 & 11.99 \\
rondrit127.tsp & 267.9 & 18.47 & 18.87 & 20.62 \\
\midrule
\multicolumn{5}{c}{probability of crossover = 0.70}\\ 
\midrule
rondrit016.tsp & 53.2 & 3.86 & 3.87 & 4.27 \\
rondrit048.tsp & 262.3 & 7.83 & 8.73 & 11.27 \\
rondrit067.tsp & 275.0 & 11.87 & 13.67 & 17.33 \\
rondrit127.tsp & 267.4 & 20.65 & 22.54 & 25.76 \\
\midrule
\multicolumn{5}{c}{probability of crossover = 0.95}\\ 
\midrule
rondrit016.tsp & 275.0 & 3.82 & 5.45 & 6.99 \\
rondrit048.tsp & 275.0 & 12.01 & 17.23 & 21.33 \\
rondrit067.tsp & 275.0 & 16.50 & 21.51 & 25.12 \\
rondrit127.tsp & 275.0 & 25.45 & 30.02 & 33.20 \\
\bottomrule 
\end{tabular} 
}
\caption{Existing genetic algorithm with varying probability of crossover.}
\label{tab:vary_crossover}
\end{table}

\begin{table}[H] 
\centering 
\makebox[\textwidth]{
\begin{tabular}{l rrrr} 
\toprule
Dataset & \# Generations & Min & Mean & Max\\ 
\midrule
\multicolumn{5}{c}{probability of mutation = 0.00}\\ 
\midrule
rondrit016.tsp & 250.4 & 3.78 & 5.04 & 6.39 \\
rondrit048.tsp & 275.0 & 11.94 & 16.97 & 21.15 \\
rondrit067.tsp & 275.0 & 16.57 & 21.50 & 24.87 \\
rondrit127.tsp & 275.0 & 25.79 & 30.13 & 33.28 \\
\midrule
\multicolumn{5}{c}{probability of mutation = 0.05}\\ 
\midrule
rondrit016.tsp & 248.6 & 3.74 & 5.13 & 6.34 \\
rondrit048.tsp & 275.0 & 11.58 & 17.10 & 21.07 \\
rondrit067.tsp & 275.0 & 16.55 & 21.47 & 25.15 \\
rondrit127.tsp & 275.0 & 25.69 & 30.07 & 33.22 \\
\midrule
\multicolumn{5}{c}{probability of mutation = 0.10}\\ 
\midrule
rondrit016.tsp & 275.0 & 3.78 & 5.41 & 6.94 \\
rondrit048.tsp & 275.0 & 11.91 & 16.94 & 20.74 \\
rondrit067.tsp & 275.0 & 16.34 & 21.52 & 24.83 \\
rondrit127.tsp & 275.0 & 25.56 & 30.13 & 33.08 \\
\midrule
\multicolumn{5}{c}{probability of mutation = 0.30}\\ 
\midrule
rondrit016.tsp & 275.0 & 3.92 & 5.71 & 7.34 \\
rondrit048.tsp & 275.0 & 11.48 & 17.30 & 21.32 \\
rondrit067.tsp & 275.0 & 15.89 & 21.31 & 25.08 \\
rondrit127.tsp & 275.0 & 25.10 & 29.95 & 32.99 \\
\midrule
\multicolumn{5}{c}{probability of mutation = 0.50}\\ 
\midrule
rondrit016.tsp & 275.0 & 3.87 & 5.90 & 7.47 \\
rondrit048.tsp & 275.0 & 11.61 & 17.31 & 21.55 \\
rondrit067.tsp & 275.0 & 15.80 & 21.55 & 25.25 \\
rondrit127.tsp & 275.0 & 24.48 & 29.71 & 32.90 \\
\midrule
\multicolumn{5}{c}{probability of mutation = 0.70}\\ 
\midrule
rondrit016.tsp & 275.0 & 3.86 & 6.01 & 7.59 \\
rondrit048.tsp & 275.0 & 11.03 & 17.27 & 21.42 \\
rondrit067.tsp & 275.0 & 15.59 & 21.50 & 25.28 \\
rondrit127.tsp & 275.0 & 25.01 & 30.03 & 32.92 \\
\midrule
\multicolumn{5}{c}{probability of mutation = 0.95}\\ 
\midrule
rondrit016.tsp & 275.0 & 3.95 & 6.12 & 7.67 \\
rondrit048.tsp & 275.0 & 11.50 & 17.51 & 21.43 \\
rondrit067.tsp & 275.0 & 15.84 & 21.66 & 25.35 \\
rondrit127.tsp & 275.0 & 24.53 & 29.79 & 33.13 \\
\bottomrule 
\end{tabular} 
}
\caption{Existing genetic algorithm with varying probability of mutation.}
\label{tab:vary_mutation}
\end{table}


\section{Code}
\lstinputlisting[language=matlab,captionpos=t,caption=The main algorithm - src/run\_ga.m]{../src/run_ga.m}
\lstinputlisting[language=matlab,captionpos=t,caption=Order crossover for task 4-  src/cross\_order.m]{../src/cross_order.m}
\lstinputlisting[language=matlab,captionpos=t,caption=High level crossover function-  src/crossover\_tsp.m]{../src/crossover_tsp.m}
\lstinputlisting[language=matlab,captionpos=t,caption=Inversion mutation for task 4 - src/mut\_inversion2.m]{../src/mut_inversion2.m}
\lstinputlisting[language=matlab,captionpos=t,caption=High level mutation function - src/mutate\_tsp.m]{../src/mutate_tsp.m}
\lstinputlisting[language=matlab,captionpos=t,caption=Fitness proportional selection for task 7a - src/sel\_fit\_prop.m]{../src/sel_fit_prop.m}
\lstinputlisting[language=matlab,captionpos=t,caption=Tournament selection for task 7a - src/sel\_tournament.m]{../src/sel_tournament.m}
\lstinputlisting[language=matlab,captionpos=t,caption=Round robin tournament survival selection for task 7b - src/sur\_sel\_rr\_tournament.m]{../src/sur_sel_rr_tournament.m}
\lstinputlisting[language=matlab,captionpos=t,caption=Template function for testing the algorithm. Other testing functions are omitted because they are very similar to this one - src/test\_template.m]{../src/test_template.m} 

\end{document}
